% !TeX program = xelatex
% !TeX encoding = UTF-8

\documentclass{beamer}

\usecolortheme{beaver}

\usepackage{fontspec}
\usepackage{xunicode}
\usepackage{mathpazo}
\defaultfontfeatures{Ligatures=TeX}
\setmainfont{TeX Gyre Pagella}
\setsansfont{Linux Biolinum O}
\setmonofont[Scale=MatchLowercase]{DejaVu Sans Mono}

\usepackage{amsmath}
%\usepackage{unicode-math}
%\setmathfont{TeX Gyre Pagella Math}

\usepackage{polyglossia}
\setdefaultlanguage[variant=american]{english}
\usepackage{csquotes}

\usepackage{hyperref}

\newcommand{\pTeX}{{\rmfamily\TeX}}
\newcommand{\pLaTeX}{{\rmfamily\LaTeX}}
\newcommand{\pLaTeXe}{{\rmfamily\LaTeXe}}

\title{Introduction to \pLaTeX}
\author{Lizhou Sha\\\href{mailto:slz@mit.edu}{slz@mit.edu}}

\begin{document}

\frame{\titlepage}

%\begin{frame}
%\frametitle{Outline}
%\tableofcontents[pausesections]
%\end{frame}

\section{Introduction}

\begin{frame}{A brief history}
\begin{itemize}
    \item \alert{\rmfamily\TeX} was created by Donald Knuth in 1977 to bring state-of-the art typography to digital printing (especially for typesetting technical materials).
        \begin{itemize}
            \item Famed for being extremely stable and bug free.
            \item Current release: 3.14159265 in January 2014.
        \end{itemize}
    \item \alert{\pLaTeX} was created by Leslie Lamport in 1985 as high-level templates on top of \pLaTeX.
        \begin{itemize}
            \item Current release: \pLaTeXe
        \end{itemize}
\end{itemize}
\pLaTeX\ is pronounced Lah-tech or Lay-tech or\ldots
\end{frame}

\begin{frame}{\pLaTeX\ is\ldots}
\ldots an advanced typesetting system
\begin{itemize}
    \item Uniform styles
    \item Structural specifications
    \item Reference tracking
    \item Bibliography
    \item Highly extensible
    \item Integrates well with scripts
\end{itemize}
\end{frame}

\begin{frame}{\pLaTeX\ is not\ldots}
\ldots a word processor, or a text editor, or a typewriter!
\begin{itemize}
    \item \pLaTeX\ takes in a text file, and dutifully produces exactly what you have specified.
    \item Use your favorite text editor, or an integrated \pLaTeX\ environment.
\end{itemize}
\begin{quotation}
You must unlearn what you have learned.
\end{quotation}
\end{frame}

\begin{frame}{Why \pLaTeX?}
\pLaTeX\ helps you separate \emph{contents} from \emph{presentation}.

\begin{columns}[c]
    \column{0.5\textwidth}
    \begin{itemize}
        \item What font size do I use to start a section?
        \item Do I italicize or small cap my abbreviations?
        \item I have to spend so much time matching the template
    \end{itemize}
    \column{0.5\textwidth}
    \begin{itemize}
        \item \texttt{\textbackslash section\{\}} \\ \ 
        \item Just \texttt{\textbackslash newcommand\{\}} \\ \ 
        \item Just swap out your \texttt{\textbackslash documentclass\{\}}
    \end{itemize}
\end{columns}

\end{frame}


\section{Wrapping Up}

\begin{frame}{Resources}
\begin{itemize}
    \item Tobias Oetiker et al. The Not So Short Introduction to {\rmfamily\LaTeXe} (\texttt{lshort}).\\
        \url{https://tobi.oetiker.ch/lshort/lshort.pdf}
    \item {\rmfamily\LaTeX} on Wikibooks\\
        \url{https://en.wikibooks.org/wiki/LaTeX}
    \item Michael Downes. Short Math Guide for {\rmfamily\LaTeX}.\\
        \url{ftp://ftp.ams.org/ams/doc/amsmath/short-math-guide.pdf}
\end{itemize}
\end{frame}

\begin{frame}{To get started}
\begin{itemize}
    \item These slides can be found at:
        \url{https://slz.scripts.mit.edu/Public/sipb/sipb-iap-latex-2016.pdf}
    \item ModernCV examples:
        \url{https://www.ctan.org/tex-archive/macros/latex/contrib/moderncv/examples}
    \item You can use this slideshow as a template of your own slides!
        \url{https://github.com/vulpicastor/sipb-iap-latex}
\end{itemize}
\end{frame}

\end{document}

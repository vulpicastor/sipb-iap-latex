% !TeX program = xelatex
% !TeX encoding = UTF-8

\documentclass{beamer}

\usepackage{fontspec}
\usepackage{xunicode}
\defaultfontfeatures{Ligatures=TeX}
\setmainfont{TeX Gyre Pagella}
\setsansfont{Linux Biolinum O}
\setmonofont[Scale=MatchLowercase]{DejaVu Sans Mono}

\usepackage{amsmath}
\usepackage{unicode-math}
\setmathfont{TeX Gyre Pagella Math}

\usepackage{polyglossia}
\setdefaultlanguage[variant=american]{english}
\usepackage{csquotes}

\usepackage{hyperref}

\title{Introduction to \rmfamily\LaTeX}
\author{Lizhou Sha\\\href{mailto:slz@mit.edu}{slz@mit.edu}}

\begin{document}

\frame{\titlepage}

\begin{frame}
\frametitle{Outline}
\tableofcontents[pausesections]
\end{frame}

\section{Introduction}

\begin{frame}{A brief history}
Donald Knuth
\end{frame}

\section{Wrapping Up}

\begin{frame}
\frametitle{Resources}
\begin{itemize}
    \item Tobias Oetiker et al. The Not So Short Introduction to {\rmfamily\LaTeX} 2$_{\epsilon}$ (\texttt{lshort}).\\
        \url{https://tobi.oetiker.ch/lshort/lshort.pdf}
    \item {\rmfamily\LaTeX} on Wikibooks\\
        \url{https://en.wikibooks.org/wiki/LaTeX}
    \item Michael Downes. Short Math Guide for {\rmfamily\LaTeX}.\\
        \url{ftp://ftp.ams.org/ams/doc/amsmath/short-math-guide.pdf}
\end{itemize}
\end{frame}

\end{document}
